\documentclass{../homework}

\title{Homework 2}
\author{}
\date{2019 September 23}

\begin{document}
\begin{problems}
\item Given a subset \(E\) of \(\RR\), define exterior measure of
  \(E\) by
  \[
    m_*(E) = \inf\set*{
      \sum_{n=1}^\infty l(I_n)
      : E \subset \bigcup_{n=1}^\infty I_n
    }
  \]
  where by \(l(I_n)\) we mean the length of \(I_n\) and the infimum is
  taken over all coverings of \(E\) by countable bounded intervals.
  Prove the following:
  \begin{problems}
  \item \(0 \le m_*(E) \le \infty\).
  \item If \(E \subset F\) then \(m_*(E) \subset m_*(F)\).
  \item \(m_*(E+x) = m_*(E)\) where \(E+x = \set{e+x : e \in E}\).
  \item \(m_*(E) = 0\) for any countable set \(E\).
  \item \(m_*(E) < \infty\) for any bounded set \(E\).
  \item
    \(m_*(E) = \inf\set*{\sum_{n=1}^\infty (b_n-a_n) : E =
      \bigcup_{n=1}^\infty (a_n, b_n)}\).
  \end{problems}

  \begin{solution}
    \begin{problems}
    \item
    \item
    \item
    \item
    \item
    \item
    \end{problems}
  \end{solution}

\item True-False: Let \(E = \QQ \cap [0, 1]\) and \(\set{I_n}\) be a
  \textbf{finite} collection of open intervals covering \(E\).  Then
  we have \(\sum_{n=1}^\infty \abs{I_n} \ge 1\).

  \begin{solution}
  \end{solution}

\item Stein and Shakarchi: page 39 numbers: 5, 6, 7, 8

  \begin{book}
    \begin{enumerate}[start=5]
    \item Suppose \(E\) is a given set, and \(\mathcal O_n\) is the
      open set:
      \[
        \mathcal O_n = \set*{x : d(x, E) < \frac 1 n}.
      \]
      Show:
      \begin{problems}
      \item If \(E\) is compact, then
        \(m(E) = \lim_{n\to\infty} m(\mathcal O_n)\).
        \label{itm:3.5.a}
      \item However, the conclusion in \ref{itm:3.5.a} may be false
        for \(E\) closed and unbounded; or \(E\) open and bounded.
      \end{problems}
    \end{enumerate}
  \end{book}

  \begin{solution}
    \begin{problems}
    \item
    \item
    \end{problems}
  \end{solution}

  \begin{book}
    \begin{enumerate}[start=6]
    \item Using translations and dilations, prove the following: Let
      \(B\) be a ball in \(\RR^d\) of radius \(r\).  Then
      \(m(B) = v_d r^d\), where \(v_d = m(B_1)\), and \(B_1\) is the
      unit ball, \(B_1 = \set{x \in \RR^d : \abs x < 1}\).

      A calculation of the constant \(v_d\) is postponed until
      Exercise 14 in the next chapter.
    \end{enumerate}
  \end{book}

  \begin{solution}
  \end{solution}

  \begin{book}
    \begin{enumerate}[start=7]
    \item If \(\delta = (\delta_1, \dots, \delta_d)\) is a \(d\)-tuple
      of positve numbers \(\delta_i > 0\), and \(E\) is a subset of
      \(\RR^d\), we define \(\delta E\) by
      \[
        \delta E = \set{
          (\delta_1 x_1, \dots, \delta_d x_d)
          : \text{where \(x_1, \dots, x_d \in E\)}
        }.
      \]
    \end{enumerate}
  \end{book}

  \begin{solution}
  \end{solution}

  \begin{book}
    \begin{enumerate}[start=8]
    \item Suppose \(L\) is a linear transformation of \(\RR^d\).  Show
      that if \(E\) is a measurable subset of \(\RR^d\), then so is
      \(L(E)\), by proceeding as follows:
      \begin{problems}
      \item Note that if \(E\) is compact, so is \(L(E)\).  Hence if
        \(E\) is an \(F_\sigma\) set, so is \(L(E)\).
      \item Because \(L\) automatically satisfies the inequality
        \[
          \abs{L(x) - L(x')} \le M \abs{x - x'}
        \]
        for some \(M\), we can see that \(L\) maps any cube of side
        length \(l\) into a cube of side length \(c_d M l\), with
        \(c_d = 2 \sqrt d\).  Now if \(m(E) = 0\), there is a
        collection of cubes \(\set{Q_j}\) such that
        \(E \subset \bigcup_j Q_j\), and \(\sum_j m(Q_j) < \epsilon\).
        Thus \(m_*(L(E)) \le c' \epsilon\), and hence \(m(L(E)) = 0\).
        Finally, use Corollary 3.5.

        One can show that \(m(L(E)) = \abs{\det L} m(E)\); see Problem
        4 in the next chapter.
      \end{problems}
    \end{enumerate}
  \end{book}

  \begin{solution}
    \begin{problems}
    \item
    \item
    \end{problems}
  \end{solution}

\item Prove the \textbf{Borel--Cantelli lemma}.

  See Stein and Shakarchi text: page 42 number: 16.

  \begin{book}
    \textbf{The Borel--Cantelli lemma}.  Suppose
    \(\set{E_k}_{k=1}^\infty\) is a countable family of measurable
    subsets of \(\RR^d\) and that
    \[
      \sum_{k=1}^\infty m(E_k) < \infty.
    \]
  \end{book}

  Remark: In probability theory the Borel--Cantelli lemma is stated as
  follows: Given \(\set{E_n}\) sequence of events in some probability
  space, if the sum of the probabilities of \(E_n\) is finite, i.e.,
  if \(\sum_{n=1}^\infty m(E_n) < \infty\) then the probability that
  infinitely many of them occur is
  \(0\), i.e., \(m(\limsup E_n) = 0\).

  \begin{solution}
  \end{solution}

\end{problems}
\end{document}