\documentclass{../homework}

\title{Homework 1}
\date{2019 September 11 (Wednesday)}
\author{}

\begin{document}

\begin{problems}
\item Let \((X, d)\) be a metric space and let \(A, B\) be two
  nonempty subsets of \(X\).  Prove that if
  \(A \cap B \ne \emptyset\), then the following inequality holds for
  the diameters \(\delta\):
  \[
    \delta(A \cup B) \le \delta(A) + \delta(B).
  \]

  Prove also that the diameter of \(A\) is equal to the diameter of
  the closure, i.e., \(\delta(A) = \delta(\overline A)\).

  Recall that \(\delta(A) = \sup\{d(x, y) : x, y \in A\}\).

  \begin{solution}
  \end{solution}

\item Let \(0<a<b\) and
  \[
    f(x) =
    \begin{cases}
      1 & \text{if \(x \in [a, b] \cap \QQ\)} \\
      0 & \text{if \(x \in [a, b]\) is irrational}.
    \end{cases}
  \]
  Find the upper and lower Riemann integrals of \(f(x)\) over
  \([a, b]\).

  \begin{solution}
  \end{solution}

\item Consider the function
  \[
    f(x) =
    \begin{cases}
      1 & 0 \le x \le 1, \\
      0 & 1 < x \le 2.
    \end{cases}
  \]
  \begin{problems}
  \item What is \(F(x) = \int_0^x f(t) \dif t\) on \([0, 2]\)?

    \begin{solution}
    \end{solution}

  \item Is \(F(x)\) continuous?

    \begin{solution}
    \end{solution}

  \item Is \(F'(x) = f(x)\)?

    \begin{solution}
    \end{solution}
  \end{problems}

\item Give examples to illustrate that
  \begin{problems}
  \item the pointwise limit of continuous (respectively,
    differentiable) functions is not necessarily continuous
    (respectively, differentiable).

    \begin{solution}
    \end{solution}

  \item the pointwise limit of integrable functions is not necessarily
    integrable.

    \begin{solution}
    \end{solution}
  \end{problems}

\item Give examples to illustrate that
  \begin{problems}
  \item there exist differentiable functions \(f_n\) and \(f\) such
    that \(f_n \to f\) pointwise on \([0, 1]\) but
    \[
      \lim_{n\to\infty} f_n'(x) \ne \prn*{\lim_{n\to\infty} f_n(x)}'
      \quad \text{when \(x=1\)},
    \]

    \begin{solution}
    \end{solution}

  \item there exist continuous functions \(f_n\) and \(f\) such that
    \(f_n \to f\) pointwise on \([0, 1]\) but
    \[
      \lim_{n\to\infty} \int_0^1 f_n(x) \dif x
      \ne \int_0^1 \prn*{\lim_{n\to\infty} f_n(x)} \dif x.
    \]

    \begin{solution}
    \end{solution}
  \end{problems}

\item Show that there exists a continuous function defined on \(\RR\)
  that is nowhere differentiable by proving the following:
  \begin{problems}
  \item Let \(g(x) = \abs x\) if \(x \in [-1, 1]\).  Extend \(g\) to
    be periodic.  Sketch \(g\) and the first few terms of the sum
    \[
      f(x) = \sum_{n=1}^\infty \prn*{\frac 3 4}^n g(4^n x).
    \]

    \begin{solution}
    \end{solution}

  \item Use the Weierstrass \(M\)-test to show that \(f\) is
    continuous.

    \begin{solution}
    \end{solution}

  \item Prove that \(f\) is not differentiable at any point.

    \begin{solution}
    \end{solution}
  \end{problems}

\item
  \begin{problems}
  \item Show that the Cantor set \(C\) can also be described as of all
    those real numbers in \([0, 1]\) which have ternary expansions
    \[
      C = \cbr*{
        x : x = \sum_{n=1}^\infty \frac{a_n}{3^n}
        \quad a_n = 0, 2
      }
    \]

    \begin{solution}
    \end{solution}

  \item Show that the length of \(C\) is equal to zero.

    \begin{solution}
    \end{solution}

  \item Show that Cantor set \(C\) can be put into one-to-one
    correspondence with the interval \([0, 1]\) and that cardinality
    of \(C\) is continuum.

    \begin{solution}
    \end{solution}

  \item Show that Cantor set is totally disconnected and perfect.

    \begin{solution}
    \end{solution}
  \end{problems}
\end{problems}
\end{document}