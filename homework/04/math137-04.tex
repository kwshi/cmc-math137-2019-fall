\documentclass{../homework}

\title{Homework 4}
\author{}
\date{2019 November 6}

\begin{document}
\begin{problems}
\item Prove Tchebychev's inequality.  Suppose \(f \ge 0\), and \(f\)
  is integrable.  If \(\alpha > 0\) and
  \(E_\alpha = \set{x : f(x) > \alpha}\), prove that
  \[
    m(E_\alpha) \le \frac 1 \alpha \int f.
  \]

  \begin{solution}
  \end{solution}

\item
  \begin{problems}
  \item Give an example of a function \(f \colon \RR \to \RR\) which
    is integrable, but \(f^2\) is not.

    \begin{solution}
    \end{solution}

  \item Show that a measurable function \(f\) is integrable if and
    only if \(\abs f\) is integrable.  Give an example of a
    nonintegrable function whose absolute value is integrable.

    \begin{solution}
    \end{solution}

  \end{problems}

\item Suppose \(f\) is Riemann integrable on the closed interval
  \([a, b]\).  Then show that \(f\) is measurable and
  \[
    (R) \int_{[a, b]} f = (L) \int_{[a, b]} f
  \]
  where the integral on the left-hand side is the Riemann integral,
  and that on the right-hand side is the Lebesgue integral.

  \begin{solution}
  \end{solution}

\item Let \(\mu\) be the counting measure on \(\NN\).  Interpret
  Fatou's lemma and the monotone and dominated convergence theorems
  about infinite series.

  \begin{solution}
  \end{solution}

\item If \(f_n, g_n, f, g\) are all integrable functions (all in
  \(L^1\)) with \(f_n \to f\) and \(g_n \to g\) a.e.,
  \(\abs{f_n} \le g_n\), and \(\int g_n \to \int g\), then show that
  \(\int f_n \to \int f\).

  Hint: Rework the proof of the dominated convergence theorem.

  \begin{solution}
  \end{solution}

\item Suppose \(f_n, f \in L^1\) and \(f_n \to f\) a.e.  Then show
  that \(\int \abs{f_n - f} \to 0\) iff
  \(\int \abs{f_n} \to \int \abs f\).

  Hint: Use Problem 5 above.

  \begin{solution}
  \end{solution}

\item Suppose \(f \ge 0\), let
  \[
    \mu(E) = \int_E f \dif m
  \]
  for a measurable set \(E\).
  \begin{problems}
  \item Show that \(\mu\) is a measure.

    \begin{solution}
    \end{solution}

  \item Show that for any \(g \ge 0\)
    \[
      \int g \dif \mu = \int f g \dif m.
    \]

    Hint: First suppose \(g\) is simple.

    \begin{solution}
    \end{solution}

  \end{problems}

\item Show that \(f(x) = \frac{\ln x}{x^2}\) is Lebesgue integrable
  over \([1, \infty)\) and that \(\int f \dif x = 1\).

  \begin{solution}
  \end{solution}

\item Show that the improper Riemann integral
  \[
    \int_0^\infty \cos(x^2) \dif x
  \]
  exists but is not Lebesgue integrable over \([0, \infty)\).

  \begin{solution}
  \end{solution}

\item Establish the Riemann--Lebesgue Theorem: If \(f\) is an
  integrable function on \((-\infty, \infty)\) then
  \[
    \lim_{n\to\infty} \int_{-\infty}^\infty f(x) \cos(nx) \dif x = 0.
  \]
  Hint: Theorem is easy if \(f\) is a simple function; then use
  Theorem 2.4 (Stein, page 71).

  \begin{solution}
  \end{solution}

\end{problems}
\end{document}