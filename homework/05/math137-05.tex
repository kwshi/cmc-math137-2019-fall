\documentclass{../homework}

\title{Homework 5}
\author{}
\date{2019 November 18}

\begin{document}
\begin{problems}
\item Let \(T\) be an arbitrary metric space and
  \(f \colon \RR \times T \to \RR\) a function.  Assume that
  \(f(\cdot, t)\) is a measurable function for each \(t \in T\) and
  \(f(x, \cdot)\) is a continuous function for each \(x \in \RR\).
  Assume also that there exists an integrable function \(g\) such that
  for each \(t \in T\) we have \(\abs{f(x, t)} \le g(x)\) for almost
  all \(x \in \RR\).  Show thatthe function \(F \colon T \to \RR\)
  defined by
  \[
    F(t) = \int_\RR f(x, t) \dif x
  \]
  is a continuous function.

  \begin{solution}
  \end{solution}

\item Let \(f\) be integrable over \((-\infty, \infty)\).
  \begin{problems}
  \item Show that
    \[
      \int f(x) \dif x = \int f(x+t) \dif x
    \]

    \begin{solution}
    \end{solution}

  \item Let \(g\) be a bounded measurable function.  Then show that
    \[
      \lim_{t\to0} \int_{-\infty}^\infty
      \abs*{g(x) \sbr*{f(x)-f(x+t)}} \dif x = 0.
    \]

    \begin{solution}
    \end{solution}

  \end{problems}

\item
  \begin{problems}
  \item Let \(\set{f_n}\) be a sequence of real valued measurable
    functions.  If \(\abs{f_n}\) converges to \(f\) in measure, show
    that \(\abs{f_n}\) is a Cauchy sequence in measure.

    \begin{solution}
    \end{solution}

  \item Show that if a sequence \(\set{f_n}\) of integrable functions
    converge to \(f\) in \(L^1\), then \(\set{f_n}\) converges to
    \(f\) in measure.  Is the converse true?

    Note: We say \(\set{f_n}\) is \emph{Cauchy in measure} if for
    every \(\epsilon > 0\),
    \[
      m\prn*{\set*{x : \abs*{f_n(x)-f_m(x)} > \epsilon}} \to 0
      \quad \text{as \(m, n \to \infty\)},
    \]
    and we say \(\set{f_n}\) \emph{converges in measure} to a
    measurable function \(f\) if for every \(\epsilon > 0\)
    \[
      m\prn*{\set*{x : \abs*{f_n(x)-f(x)} > \epsilon}} \to 0
      \quad \text{as \(n \to \infty\)}.
    \]

    \begin{solution}
    \end{solution}

  \end{problems}

\item Compute the following limits and justify the calculations:
  \begin{problems}
  \item
    \(\displaystyle\lim_{n\to\infty} \int_0^\infty \sbr*{1+\frac x
      n}^{-n} \sin\frac x n \dif x\)

    \begin{solution}
    \end{solution}

  \item
    \(\displaystyle\lim_{n\to\infty} \int_0^\infty n \sin\frac x n
    \sbr*{x \prn*{1+x^2}}^{-1} \dif x\)

    \begin{solution}
    \end{solution}

  \item
    \(\displaystyle\lim_{n\to\infty} \int_a^\infty n \prn*{1+n^2
      x^2}^{-1} \dif x\)

    \begin{solution}
    \end{solution}

  \end{problems}

\item Show that
  \(\int_0^\infty x^{2n} e^{-x^2} \dif x = \frac{(2n)!}{2^{2n} n!}
  \cdot \frac{\sqrt \pi}{2}\) for \(n = 0, 1, 2, \dots\).

  Hint: Use induction on \(n\) and the fact that
  \(\int_0^\infty e^{-x^2} \dif x = \frac{\sqrt \pi}{2}\) (Euler's
  formula).

  \begin{solution}
  \end{solution}

\item Show that for \(a > 0\),
  \(\int_{-\infty}^\infty e^{-x^2} \cos(ax) \dif x = \sqrt \pi
  e^{-\frac{a^2}{4}}\).

  Hint: Use problem 5 above.

  \begin{solution}
  \end{solution}

\item Suppose that \(f\) is a real, continuously differentiable
  function on \([a, b]\), that \(f(a) = f(b) = 0\), and that
  \(\int_a^b f^2(x) \dif x = 1\).  Prove that:
  \begin{problems}
  \item \(\displaystyle\int_a^b x f(x) f'(x) \dif x = -\frac 1 2\)

    \begin{solution}
    \end{solution}

  \item
    \(\displaystyle\int_a^b \sbr*{f'(x)}^2 \int_a^b x^2 f^2(x) \dif x
    \ge \frac 1 4\)

    Hint: recall that \(\abr{f, g} = \int_a^b f(x) g(x) \dif x\)
    defines an inner product on \(C[a, b]\).

    \begin{solution}
    \end{solution}

  \end{problems}

\end{problems}
\end{document}